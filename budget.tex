\section*{Budget Justification and Allocation}

The total requested budget for the CausalPCa project is \textbf{\EUR{3,243,919}} over a period of 36 months. This budget is the result of a rigorous planning process to ensure all necessary resources are allocated to achieve the ambitious objectives outlined in this proposal, while delivering maximum value for money. The costs are broken down by Work Package (WP) and cost category, with detailed justifications provided below. All costs are estimated in EUR.

\subsection*{A. Direct Costs}

\subsubsection*{A.1 Personnel Costs}
Personnel costs are the most significant part of the budget, reflecting the highly skilled, interdisciplinary team required for this project. Integrating advanced imaging, clinical data, and multi-omics data requires a diverse team with expertise in nuclear medicine, radiology, data science, mathematics, and software engineering. Costs are calculated for a 36-month project duration based on institutional salary tables (TV-L), including all social security contributions. A total of \textbf{282 person-months (PMs)} are budgeted.

\begin{itemize}
    \item \textbf{Scientific Staff (Wissenschaftlicher Mitarbeiter) - PI (1 FTE, 36 PMs):} \EUR{263,141}
    \item \textbf{Scientific Staff (Wissenschaftlicher Mitarbeiter) - PhD Student (3 FTEs, 108 PMs):} \EUR{789,423}
    \item \textbf{Scientific Staff (Wissenschaftlicher Mitarbeiter) - Mathematician (0.5 FTE, 18 PMs):} \EUR{131,571}
    \item \textbf{Scientific Staff (Wissenschaftlicher Mitarbeiter) - Data Scientist (1 FTE, 36 PMs):} \EUR{263,141}
    \item \textbf{Technical Staff (Technischer Angestellter) - Programmer (2 FTEs, 72 PMs):} \EUR{405,966}
    \item \textbf{Technical Staff (Technischer Angestellter) - Project Manager (0.5 FTE, 18 PMs):} \EUR{101,492}
    \item \textbf{Student/Research Assistant (Wissenschaftliche Hilfskraft) (40hrs/month, 36 months):} \EUR{33,040}
\end{itemize}

\textbf{Total Estimated Personnel Costs: \EUR{1,987,774}}


\subsubsection*{A.2 Equipment Costs}
The requested budget of \EUR{536,000} for computational resources represents a strategic, one-time capital investment in the core enabling technology of this project. This on-premise infrastructure is not an operational overhead but the central scientific instrument—the \textbf{primary discovery engine}—required to achieve the project's groundbreaking ambition. The amount is based on a detailed analysis of the project's workload and market pricing for state-of-the-art AI hardware.

\begin{table}[H]
\centering
\caption{The Revised Computational Resource Budget (3 Years)}
\label{tab:computational_budget}
\sisetup{group-separator={,}}
\begin{tabular}{lS[table-format=6.0]S[table-format=6.0]S[table-format=6.0]S[table-format=6.0]}
\toprule
\textbf{Cost Category} & {\textbf{Year 1 (€)}} & {\textbf{Year 2 (€)}} & {\textbf{Year 3 (€)}} & {\textbf{Total (€)}} \\
\midrule
NVIDIA DGX B200 System & 489000 & 0 & 0 & 489000 \\
300 TB Storage Solution & 28000 & 0 & 0 & 28000 \\
Installation \& Integration & 19000 & 0 & 0 & 19000 \\
\midrule
\textbf{TOTAL} & \textbf{536000} & \textbf{0} & \textbf{0} & \textbf{536000} \\
\bottomrule
\end{tabular}
\end{table}

\paragraph{1. Strategic Imperative: An NVIDIA DGX B200 as the Project's Cornerstone}
To pursue the project's visionary goal of creating a causal AI framework for medicine, an investment in computational power that is commensurate with the scale of the scientific challenge is required. The budget is therefore centered on the acquisition of an \textbf{NVIDIA DGX B200 system}. This is not a conventional server but a purpose-built, integrated AI supercomputer that directly addresses the primary technical risks of the project.

\paragraph{2. Justification for the DGX B200 Solution}
The choice of the DGX B200 is driven by key technical requirements of the proposed research:
\begin{itemize}
    \item \textbf{Solving the VRAM Bottleneck:} The project's focus on high-resolution 3D medical imaging is severely limited by GPU memory (VRAM). The DGX B200 provides a massive \textbf{1.44 TB of unified, high-bandwidth GPU memory}, which is essential for training the large, complex 3D VAE and Diffusion Models at the heart of our methodology. This capability is critical to de-risk the core of the project.
    \item \textbf{Powering Complex, Multi-Stage Workflows:} Our four-stage causal framework requires intensive, iterative re-training cycles. The DGX B200's architecture, with eight tightly interconnected GPUs, is engineered for exactly this kind of complex, distributed workload, ensuring maximum efficiency for model development.
    \item \textbf{Turnkey Solution to Reduce Engineering Overhead:} As a fully integrated and validated platform, the DGX B200 allows the research team to focus immediately on scientific discovery rather than spending months on system integration and debugging. It includes the necessary NVIDIA AI Enterprise software and support, accelerating the timeline from deployment to discovery.
    \item \textbf{High-Throughput Data Processing:} The system's architecture, including features like GPUDirect Storage, is optimized to handle the large-scale (~200-400 TB) multimodal datasets of this project, ensuring the powerful GPUs are never left "starved" for data.
\end{itemize}

\paragraph{3. Storage and Integration}
To complement the DGX B200, the budget includes a \textbf{300 TB enterprise storage solution} and the mandatory \textbf{professional installation and integration services}. This ensures the entire system is deployed correctly, validated, and ready to support the project's ambitious research agenda from day one. This on-premise solution provides a cost-effective and secure way to manage the project's large data assets without the variable ongoing costs of cloud storage. Should market conditions allow for the procurement of this hardware at a lower cost than budgeted, any remaining funds will be re-allocated to secure cloud solutions to further enhance the project's computational flexibility.

\textbf{Total Estimated Equipment Costs: \EUR{536,000}}

\subsubsection*{A.3 Travel Costs}
The travel budget is allocated for project meetings and dissemination of project results at leading international conferences.
\begin{itemize}
    \item \textbf{Project Meetings (3):} Budget for travel and accommodation for the project team and key external collaborators to attend a kick-off meeting (M1 in Magdeburg), a mid-term review meeting (M18), and a final project meeting (M36 in Magdeburg).
    \item \textbf{International Conferences (6):} Budget to allow key personnel (PI, PhDs, Programmers) to present project findings at major international conferences such as RSNA, ECR, MICCAI, and NeurIPS. This is crucial for WP6 (Dissemination).
\end{itemize}

\textbf{Total Estimated Travel Costs: \EUR{32,500}}

\subsubsection*{A.4 Other Direct Costs}
This category includes costs for publications, dataset access, and other minor expenses.
\begin{itemize}
    \item \textbf{Open Access Publication Fees:} To comply with Horizon Europe's open science mandate, we have budgeted for Article Processing Charges (APCs) for an estimated 10 high-impact journal publications.
    \item \textbf{UK Biobank Access:} A fee of \EUR{12,000} is budgeted for access to the UK Biobank dataset.
    \item \textbf{Software Licenses:} A provision of \EUR{36,000} is budgeted for specialized software licenses.
    \item \textbf{Clinical Consultations:} A budget of \EUR{54,000} is allocated for small monthly consultations with external experts in urology, histopathology, and radiology.
\end{itemize}

\textbf{Total Estimated Other Direct Costs: \EUR{112,000}}

\subsection*{B. Indirect Costs (Overheads)}
Indirect costs are calculated as a flat rate of \textbf{20\% of the total direct costs}, in accordance with Horizon Europe rules. These costs cover general institutional overheads.

\textbf{Total Direct Costs (A): \EUR{2,668,274}}

\textbf{Indirect Costs (B = 20\% of A): \EUR{533,655}}

\subsection*{C. Total Project Budget}
The total requested funding is the sum of total direct and indirect costs.

\textbf{Total Estimated Project Cost (A + B):} \EUR{3,201,929}

\subsection*{Budget Allocation per Work Package (in EUR)}

\begin{table}[H]
\centering
\caption{Estimated Budget Allocation per Work Package}
\label{tab:budget_wp}
\sisetup{group-separator={,}}
\begin{tabular}{lS[table-format=7.0]S[table-format=6.0]S[table-format=7.0]}
\toprule
\textbf{Work Package} & {\textbf{Direct Costs}} & {\textbf{Indirect Costs}} & {\textbf{Total Cost}} \\
\midrule
WP1: Data Curation & 450000 & 90000 & 540000 \\
WP2: Supervisor Models & 523000 & 104600 & 627600 \\
WP3: Causal VAE & 583000 & 116600 & 699600 \\
WP4: Temporal Modeling & 523000 & 104600 & 627600 \\
WP5: Validation & 200000 & 40000 & 240000 \\
WP6: Dissemination & 110000 & 22000 & 132000 \\
WP7: Project Management & 279274 & 55855 & 335129 \\
\midrule
\textbf{Total} & {\textbf{2668274}} & {\textbf{533655}} & {\textbf{3201929}} \\
\bottomrule
\end{tabular}
\end{table}